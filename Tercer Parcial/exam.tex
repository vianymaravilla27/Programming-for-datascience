% Options for packages loaded elsewhere
\PassOptionsToPackage{unicode}{hyperref}
\PassOptionsToPackage{hyphens}{url}
%
\documentclass[
]{article}
\usepackage{amsmath,amssymb}
\usepackage{lmodern}
\usepackage{ifxetex,ifluatex}
\ifnum 0\ifxetex 1\fi\ifluatex 1\fi=0 % if pdftex
  \usepackage[T1]{fontenc}
  \usepackage[utf8]{inputenc}
  \usepackage{textcomp} % provide euro and other symbols
\else % if luatex or xetex
  \usepackage{unicode-math}
  \defaultfontfeatures{Scale=MatchLowercase}
  \defaultfontfeatures[\rmfamily]{Ligatures=TeX,Scale=1}
\fi
% Use upquote if available, for straight quotes in verbatim environments
\IfFileExists{upquote.sty}{\usepackage{upquote}}{}
\IfFileExists{microtype.sty}{% use microtype if available
  \usepackage[]{microtype}
  \UseMicrotypeSet[protrusion]{basicmath} % disable protrusion for tt fonts
}{}
\makeatletter
\@ifundefined{KOMAClassName}{% if non-KOMA class
  \IfFileExists{parskip.sty}{%
    \usepackage{parskip}
  }{% else
    \setlength{\parindent}{0pt}
    \setlength{\parskip}{6pt plus 2pt minus 1pt}}
}{% if KOMA class
  \KOMAoptions{parskip=half}}
\makeatother
\usepackage{xcolor}
\IfFileExists{xurl.sty}{\usepackage{xurl}}{} % add URL line breaks if available
\IfFileExists{bookmark.sty}{\usepackage{bookmark}}{\usepackage{hyperref}}
\hypersetup{
  pdftitle={Examen},
  pdfauthor={Salazar Vega Rodrigo},
  hidelinks,
  pdfcreator={LaTeX via pandoc}}
\urlstyle{same} % disable monospaced font for URLs
\usepackage[margin=1in]{geometry}
\usepackage{color}
\usepackage{fancyvrb}
\newcommand{\VerbBar}{|}
\newcommand{\VERB}{\Verb[commandchars=\\\{\}]}
\DefineVerbatimEnvironment{Highlighting}{Verbatim}{commandchars=\\\{\}}
% Add ',fontsize=\small' for more characters per line
\usepackage{framed}
\definecolor{shadecolor}{RGB}{248,248,248}
\newenvironment{Shaded}{\begin{snugshade}}{\end{snugshade}}
\newcommand{\AlertTok}[1]{\textcolor[rgb]{0.94,0.16,0.16}{#1}}
\newcommand{\AnnotationTok}[1]{\textcolor[rgb]{0.56,0.35,0.01}{\textbf{\textit{#1}}}}
\newcommand{\AttributeTok}[1]{\textcolor[rgb]{0.77,0.63,0.00}{#1}}
\newcommand{\BaseNTok}[1]{\textcolor[rgb]{0.00,0.00,0.81}{#1}}
\newcommand{\BuiltInTok}[1]{#1}
\newcommand{\CharTok}[1]{\textcolor[rgb]{0.31,0.60,0.02}{#1}}
\newcommand{\CommentTok}[1]{\textcolor[rgb]{0.56,0.35,0.01}{\textit{#1}}}
\newcommand{\CommentVarTok}[1]{\textcolor[rgb]{0.56,0.35,0.01}{\textbf{\textit{#1}}}}
\newcommand{\ConstantTok}[1]{\textcolor[rgb]{0.00,0.00,0.00}{#1}}
\newcommand{\ControlFlowTok}[1]{\textcolor[rgb]{0.13,0.29,0.53}{\textbf{#1}}}
\newcommand{\DataTypeTok}[1]{\textcolor[rgb]{0.13,0.29,0.53}{#1}}
\newcommand{\DecValTok}[1]{\textcolor[rgb]{0.00,0.00,0.81}{#1}}
\newcommand{\DocumentationTok}[1]{\textcolor[rgb]{0.56,0.35,0.01}{\textbf{\textit{#1}}}}
\newcommand{\ErrorTok}[1]{\textcolor[rgb]{0.64,0.00,0.00}{\textbf{#1}}}
\newcommand{\ExtensionTok}[1]{#1}
\newcommand{\FloatTok}[1]{\textcolor[rgb]{0.00,0.00,0.81}{#1}}
\newcommand{\FunctionTok}[1]{\textcolor[rgb]{0.00,0.00,0.00}{#1}}
\newcommand{\ImportTok}[1]{#1}
\newcommand{\InformationTok}[1]{\textcolor[rgb]{0.56,0.35,0.01}{\textbf{\textit{#1}}}}
\newcommand{\KeywordTok}[1]{\textcolor[rgb]{0.13,0.29,0.53}{\textbf{#1}}}
\newcommand{\NormalTok}[1]{#1}
\newcommand{\OperatorTok}[1]{\textcolor[rgb]{0.81,0.36,0.00}{\textbf{#1}}}
\newcommand{\OtherTok}[1]{\textcolor[rgb]{0.56,0.35,0.01}{#1}}
\newcommand{\PreprocessorTok}[1]{\textcolor[rgb]{0.56,0.35,0.01}{\textit{#1}}}
\newcommand{\RegionMarkerTok}[1]{#1}
\newcommand{\SpecialCharTok}[1]{\textcolor[rgb]{0.00,0.00,0.00}{#1}}
\newcommand{\SpecialStringTok}[1]{\textcolor[rgb]{0.31,0.60,0.02}{#1}}
\newcommand{\StringTok}[1]{\textcolor[rgb]{0.31,0.60,0.02}{#1}}
\newcommand{\VariableTok}[1]{\textcolor[rgb]{0.00,0.00,0.00}{#1}}
\newcommand{\VerbatimStringTok}[1]{\textcolor[rgb]{0.31,0.60,0.02}{#1}}
\newcommand{\WarningTok}[1]{\textcolor[rgb]{0.56,0.35,0.01}{\textbf{\textit{#1}}}}
\usepackage{graphicx}
\makeatletter
\def\maxwidth{\ifdim\Gin@nat@width>\linewidth\linewidth\else\Gin@nat@width\fi}
\def\maxheight{\ifdim\Gin@nat@height>\textheight\textheight\else\Gin@nat@height\fi}
\makeatother
% Scale images if necessary, so that they will not overflow the page
% margins by default, and it is still possible to overwrite the defaults
% using explicit options in \includegraphics[width, height, ...]{}
\setkeys{Gin}{width=\maxwidth,height=\maxheight,keepaspectratio}
% Set default figure placement to htbp
\makeatletter
\def\fps@figure{htbp}
\makeatother
\setlength{\emergencystretch}{3em} % prevent overfull lines
\providecommand{\tightlist}{%
  \setlength{\itemsep}{0pt}\setlength{\parskip}{0pt}}
\setcounter{secnumdepth}{-\maxdimen} % remove section numbering
\ifluatex
  \usepackage{selnolig}  % disable illegal ligatures
\fi

\title{Examen}
\author{Salazar Vega Rodrigo}
\date{14/12/2021}

\begin{document}
\maketitle

\begin{Shaded}
\begin{Highlighting}[]
\CommentTok{\#knitr::opts\_chunk$set(echo = TRUE)}
\FunctionTok{library}\NormalTok{(nortest)}
\FunctionTok{library}\NormalTok{(factoextra)}
\end{Highlighting}
\end{Shaded}

\begin{verbatim}
## Warning: package 'factoextra' was built under R version 4.1.2
\end{verbatim}

\begin{verbatim}
## Loading required package: ggplot2
\end{verbatim}

\begin{verbatim}
## Welcome! Want to learn more? See two factoextra-related books at https://goo.gl/ve3WBa
\end{verbatim}

\begin{Shaded}
\begin{Highlighting}[]
\FunctionTok{library}\NormalTok{(ggplot2)}
\FunctionTok{library}\NormalTok{(polycor)}
\end{Highlighting}
\end{Shaded}

\begin{verbatim}
## Warning: package 'polycor' was built under R version 4.1.2
\end{verbatim}

\begin{Shaded}
\begin{Highlighting}[]
\FunctionTok{library}\NormalTok{(ggcorrplot)}
\FunctionTok{library}\NormalTok{(psych)}
\end{Highlighting}
\end{Shaded}

\begin{verbatim}
## Warning: package 'psych' was built under R version 4.1.2
\end{verbatim}

\begin{verbatim}
## 
## Attaching package: 'psych'
\end{verbatim}

\begin{verbatim}
## The following object is masked from 'package:polycor':
## 
##     polyserial
\end{verbatim}

\begin{verbatim}
## The following objects are masked from 'package:ggplot2':
## 
##     %+%, alpha
\end{verbatim}

\begin{Shaded}
\begin{Highlighting}[]
\NormalTok{datos }\OtherTok{\textless{}{-}} \FunctionTok{read.table}\NormalTok{(}\FunctionTok{file.choose}\NormalTok{(), }\AttributeTok{header=}\ConstantTok{TRUE}\NormalTok{)}
\NormalTok{datos }\OtherTok{\textless{}{-}} \FunctionTok{data.frame}\NormalTok{(datos)}
\NormalTok{datos\_esc }\OtherTok{\textless{}{-}} \FunctionTok{scale}\NormalTok{(datos)}
\end{Highlighting}
\end{Shaded}

\hypertarget{resolucion-de-examen.}{%
\subsection{Resolucion de examen.}\label{resolucion-de-examen.}}

Primero cargamos los datos al entorno de trabajo y lo transformamos a un
data frame para poder trabajar más comodo don dichos datos, empezaremos
con la prueba de hipotesis, ya que en esta podemos manipular los datos
tal cual esta, en las demás pruebas es necesario escalar los datos, por
eso empezaremos con esta, aplicaremos una prueba de normalidad.

\begin{Shaded}
\begin{Highlighting}[]
  \FunctionTok{cat}\NormalTok{(}\StringTok{"Exploracion de datos para ver si tiene una distribucion normal}\SpecialCharTok{\textbackslash{}n}\StringTok{"}\NormalTok{)}
\end{Highlighting}
\end{Shaded}

\begin{verbatim}
## Exploracion de datos para ver si tiene una distribucion normal
\end{verbatim}

\begin{Shaded}
\begin{Highlighting}[]
  \FunctionTok{print}\NormalTok{(}\StringTok{"Anderson{-}Darling normality test}\SpecialCharTok{\textbackslash{}n}\StringTok{"}\NormalTok{)}
\end{Highlighting}
\end{Shaded}

\begin{verbatim}
## [1] "Anderson-Darling normality test\n"
\end{verbatim}

\begin{Shaded}
\begin{Highlighting}[]
  \FunctionTok{print}\NormalTok{(}\FunctionTok{ad.test}\NormalTok{(datos}\SpecialCharTok{$}\NormalTok{laufkont))}
\end{Highlighting}
\end{Shaded}

\begin{verbatim}
## 
##  Anderson-Darling normality test
## 
## data:  datos$laufkont
## A = 86.287, p-value < 2.2e-16
\end{verbatim}

\begin{Shaded}
\begin{Highlighting}[]
  \FunctionTok{cat}\NormalTok{(}\StringTok{"}\SpecialCharTok{\textbackslash{}n}\StringTok{Como el p{-}value es menor a 0.5 podemos descartar que tenga una distribucion normal"}\NormalTok{)}
\end{Highlighting}
\end{Shaded}

\begin{verbatim}
## 
## Como el p-value es menor a 0.5 podemos descartar que tenga una distribucion normal
\end{verbatim}

\begin{Shaded}
\begin{Highlighting}[]
  \FunctionTok{print}\NormalTok{(}\FunctionTok{t.test}\NormalTok{(datos}\SpecialCharTok{$}\NormalTok{laufkont, }\AttributeTok{alternative =} \StringTok{"less"}\NormalTok{, }\AttributeTok{conf.level =} \FloatTok{0.95}\NormalTok{))}
\end{Highlighting}
\end{Shaded}

\begin{verbatim}
## 
##  One Sample t-test
## 
## data:  datos$laufkont
## t = 64.798, df = 999, p-value = 1
## alternative hypothesis: true mean is less than 0
## 95 percent confidence interval:
##      -Inf 2.642477
## sample estimates:
## mean of x 
##     2.577
\end{verbatim}

\hypertarget{analisis-factorial.}{%
\subsection{Analisis factorial.}\label{analisis-factorial.}}

Como se habia mencionado anteriormente para trabajar esta técnica
tenemos que tener los datos escalados, por lo se realiza eso y
procedemos a realizar el análisis, utilizamos una funcion la que nos
muestra cual es el número de factores optimos para trabajar con nuestro
data set, el resultado son los siguientes:

\begin{Shaded}
\begin{Highlighting}[]
\CommentTok{\# Obtenemos la matriz de correlacion policorica}
\NormalTok{  mat\_cor }\OtherTok{\textless{}{-}} \FunctionTok{hetcor}\NormalTok{(datos\_esc)}\SpecialCharTok{$}\NormalTok{correlations }\CommentTok{\#matriz de correlacion policorica}
  \FunctionTok{plot}\NormalTok{(}\FunctionTok{ggcorrplot}\NormalTok{(mat\_cor, }\AttributeTok{type=}\StringTok{"lower"}\NormalTok{, }\AttributeTok{hc.order =}\NormalTok{ T, }\AttributeTok{title =} \StringTok{"Grafica de correlaciones"}\NormalTok{))}
\end{Highlighting}
\end{Shaded}

\includegraphics{exam_files/figure-latex/pressure2-1.pdf}

\begin{Shaded}
\begin{Highlighting}[]
  \CommentTok{\# Verificamos que la matriz sea factoriazble}
  \FunctionTok{cortest.bartlett}\NormalTok{(mat\_cor, }\AttributeTok{n =} \DecValTok{100}\NormalTok{)}\OtherTok{{-}\textgreater{}}\NormalTok{p\_esf}
  \FunctionTok{cat}\NormalTok{(}\StringTok{"}\SpecialCharTok{\textbackslash{}n}\StringTok{Bartlett Test}\SpecialCharTok{\textbackslash{}n}\StringTok{"}\NormalTok{)}
\end{Highlighting}
\end{Shaded}

\begin{verbatim}
## 
## Bartlett Test
\end{verbatim}

\begin{Shaded}
\begin{Highlighting}[]
  \FunctionTok{print}\NormalTok{(p\_esf}\SpecialCharTok{$}\NormalTok{p)}
\end{Highlighting}
\end{Shaded}

\begin{verbatim}
## [1] 0.02134718
\end{verbatim}

\begin{Shaded}
\begin{Highlighting}[]
  \FunctionTok{cat}\NormalTok{(}\StringTok{"}\SpecialCharTok{\textbackslash{}n}\StringTok{KMO}\SpecialCharTok{\textbackslash{}n}\StringTok{"}\NormalTok{)}
\end{Highlighting}
\end{Shaded}

\begin{verbatim}
## 
## KMO
\end{verbatim}

\begin{Shaded}
\begin{Highlighting}[]
  \FunctionTok{print}\NormalTok{(}\FunctionTok{KMO}\NormalTok{(mat\_cor))}
\end{Highlighting}
\end{Shaded}

\begin{verbatim}
## Kaiser-Meyer-Olkin factor adequacy
## Call: KMO(r = mat_cor)
## Overall MSA =  0.6
## MSA for each item = 
## laufkont laufzeit    moral     verw    hoehe sparkont  beszeit     rate 
##     0.65     0.57     0.58     0.53     0.54     0.63     0.70     0.32 
##   famges   buerge wohnzeit     verm    alter weitkred     wohn bishkred 
##     0.54     0.60     0.57     0.73     0.61     0.57     0.60     0.53 
##    beruf     pers    telef  gastarb   kredit 
##     0.66     0.61     0.70     0.66     0.66
\end{verbatim}

\begin{Shaded}
\begin{Highlighting}[]
  \CommentTok{\# Determinar el numero de factores}
  \FunctionTok{plot}\NormalTok{(}\FunctionTok{scree}\NormalTok{(mat\_cor))}
\end{Highlighting}
\end{Shaded}

\includegraphics{exam_files/figure-latex/pressure2-2.pdf}

\begin{Shaded}
\begin{Highlighting}[]
  \FunctionTok{plot}\NormalTok{(}\FunctionTok{fa.parallel}\NormalTok{(mat\_cor,}\AttributeTok{n.obs=}\DecValTok{200}\NormalTok{,}\AttributeTok{fa=}\StringTok{"fa"}\NormalTok{,}\AttributeTok{fm=}\StringTok{"minres"}\NormalTok{))}
\end{Highlighting}
\end{Shaded}

\includegraphics{exam_files/figure-latex/pressure2-3.pdf}

\begin{verbatim}
## Parallel analysis suggests that the number of factors =  4  and the number of components =  NA
\end{verbatim}

\includegraphics{exam_files/figure-latex/pressure2-4.pdf}

\begin{Shaded}
\begin{Highlighting}[]
  \CommentTok{\#Rotacion}
\NormalTok{  rot}\OtherTok{\textless{}{-}}\FunctionTok{c}\NormalTok{(}\StringTok{"varimax"}\NormalTok{)}
  
\NormalTok{  bi\_mod}\OtherTok{\textless{}{-}}\ControlFlowTok{function}\NormalTok{(tipo)\{}
    \FunctionTok{biplot.psych}\NormalTok{(}\FunctionTok{fa}\NormalTok{(datos\_esc, }\AttributeTok{nfactors =} \DecValTok{4}\NormalTok{, }\AttributeTok{fm =} \StringTok{"minres"}\NormalTok{, }\AttributeTok{rotate =}\NormalTok{ tipo),}\AttributeTok{main =} \FunctionTok{paste}\NormalTok{(}\StringTok{"Biplot con rotación"}\NormalTok{,tipo),}\AttributeTok{col=}\FunctionTok{c}\NormalTok{(}\DecValTok{2}\NormalTok{,}\DecValTok{3}\NormalTok{,}\DecValTok{4}\NormalTok{),}\AttributeTok{pch =} \FunctionTok{c}\NormalTok{(}\DecValTok{21}\NormalTok{,}\DecValTok{18}\NormalTok{))  }
\NormalTok{  \}}
  \FunctionTok{sapply}\NormalTok{(rot,bi\_mod)}
\end{Highlighting}
\end{Shaded}

\begin{verbatim}
## Warning in fac(r = r, nfactors = nfactors, n.obs = n.obs, rotate = rotate, : An
## ultra-Heywood case was detected. Examine the results carefully
\end{verbatim}

\includegraphics{exam_files/figure-latex/pressure2-5.pdf}

\begin{verbatim}
## $varimax
## NULL
\end{verbatim}

\begin{Shaded}
\begin{Highlighting}[]
  \CommentTok{\# Interpretacion}
\NormalTok{  modelo\_varimax}\OtherTok{\textless{}{-}}\FunctionTok{fa}\NormalTok{(mat\_cor,}\AttributeTok{nfactors =} \DecValTok{4}\NormalTok{,}\AttributeTok{rotate =} \StringTok{"varimax"}\NormalTok{,}
                     \AttributeTok{fa=}\StringTok{"minres"}\NormalTok{)}
\end{Highlighting}
\end{Shaded}

\begin{verbatim}
## Warning in fac(r = r, nfactors = nfactors, n.obs = n.obs, rotate = rotate, : An
## ultra-Heywood case was detected. Examine the results carefully
\end{verbatim}

\begin{Shaded}
\begin{Highlighting}[]
  \FunctionTok{fa.diagram}\NormalTok{(modelo\_varimax, }\AttributeTok{main =} \StringTok{"Gráfico AF"}\NormalTok{)}
\end{Highlighting}
\end{Shaded}

\includegraphics{exam_files/figure-latex/pressure2-6.pdf}

\begin{Shaded}
\begin{Highlighting}[]
  \FunctionTok{print}\NormalTok{(modelo\_varimax}\SpecialCharTok{$}\NormalTok{loadings,}\AttributeTok{cut=}\DecValTok{0}\NormalTok{) }
\end{Highlighting}
\end{Shaded}

\begin{verbatim}
## 
## Loadings:
##          MR1    MR2    MR3    MR4   
## laufkont  0.080  0.057  0.567 -0.019
## laufzeit  0.518  0.016 -0.211  0.316
## moral    -0.082  0.345  0.363  0.022
## verw      0.136 -0.026 -0.034  0.002
## hoehe     0.528  0.033 -0.126  0.855
## sparkont  0.108  0.089  0.265  0.032
## beszeit   0.120  0.419  0.112 -0.089
## rate      0.166  0.104 -0.044 -0.390
## famges   -0.019  0.154  0.036 -0.017
## buerge   -0.205 -0.012 -0.111  0.073
## wohnzeit  0.090  0.327 -0.028 -0.046
## verm      0.533  0.180 -0.179  0.016
## alter     0.064  0.556  0.039 -0.040
## weitkred -0.059 -0.073  0.207 -0.031
## wohn      0.275  0.360 -0.088 -0.034
## bishkred -0.059  0.375  0.134  0.069
## beruf     0.521 -0.006  0.078  0.019
## pers      0.129 -0.276  0.083 -0.097
## telef     0.423  0.119  0.133  0.087
## gastarb   0.279 -0.033 -0.024 -0.134
## kredit   -0.139  0.120  0.571 -0.007
## 
##                  MR1   MR2   MR3   MR4
## SS loadings    1.619 1.173 1.073 1.045
## Proportion Var 0.077 0.056 0.051 0.050
## Cumulative Var 0.077 0.133 0.184 0.234
\end{verbatim}

Obtenemos la matriz de correlaciones, asi como el grafico donde muestra
el número de factores requeridos y la relación que tiene las variables
con cada factor.

\hypertarget{k-means}{%
\subsection{k-means}\label{k-means}}

Para trabajar con el k-means nos guiamos un poco con el analisis
anterior, definimos 3 grupos para trabajar con este análisis y obtenemos
el siguiente resultado.

\begin{Shaded}
\begin{Highlighting}[]
\CommentTok{\#se aplica el algoritmo k{-}means}
\NormalTok{  grupos }\OtherTok{\textless{}{-}} \FunctionTok{kmeans}\NormalTok{(datos\_esc, }\AttributeTok{centers =} \DecValTok{3}\NormalTok{, }\AttributeTok{nstart =} \DecValTok{25}\NormalTok{)}
  
  \CommentTok{\#Graficar los grupos}
  \FunctionTok{plot}\NormalTok{(}\FunctionTok{fviz\_cluster}\NormalTok{(grupos, }\AttributeTok{data =}\NormalTok{ datos\_esc, }\AttributeTok{main =} \StringTok{"Clustering K medias"}\NormalTok{))}
\end{Highlighting}
\end{Shaded}

\includegraphics{exam_files/figure-latex/pressure1-1.pdf}

\begin{Shaded}
\begin{Highlighting}[]
\NormalTok{  relaciones }\OtherTok{\textless{}{-}} \FunctionTok{data.frame}\NormalTok{(}\AttributeTok{grupos\_km =}\NormalTok{ grupos}\SpecialCharTok{$}\NormalTok{cluster)}
\end{Highlighting}
\end{Shaded}


\end{document}
